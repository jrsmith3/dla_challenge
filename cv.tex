% \section{Joshua Ryan Smith}

% \href{mailto:joshua.r.smith@gmail.com}{joshua.r.smith@gmail.com}\\(919)
% 413.0396\\\href{http://orcid.org/0000-0002-3137-7180}{ORCID}\\\href{http://github.com/jrsmith3}{github}

\section{Summary}

I am a physicist transitioning from basic scientific research to industry in search of new challenges.
I have over a decade of experience developing software for scientific computation, managing the technical and strategic direction of both computational and experimental projects in the fields of semiconductor and energy research, and communicating the results to audiences of diverse backgrounds.

\section{Skills and capabilities}

My specialty is developing mathematical models of physical systems and implementing them in software using industry-standard software development practices.
I have experience with the version control systems subversion and git, the programming languages python and MATLAB, SQLite databases, continuous integration with Travis-CI, and project management with GitHub.

\section{Experience}

\textbf{US Army Research Laboratory} - ORAU Senior Researcher, Sensors
and Electron Devices Directorate (August 2011 - Present).

At ARL I have investigated radioisotope batteries, namely betavoltaic
and betaphotovoltaic devices. Our team demonstrated a working GaN based
betavoltaic device -- I specified the structure, verified the devices
using electron beam induced current scanning tunneling microscopy, and
guided the devices through the various fabrication and characterization
steps. For the betaphotovoltaic part of the project, I specified the
AlGaN photovoltaic structure and developed a process of electrophoretic
deposition to apply phosphor to the photovoltaic devices
post-fabrication.

In addition to the radioisotope batteries, I developed a model of
electron transport through a thermoelectron energy conversion device
featuring a negative electron affinity anode. The software I wrote for
this project is used by university research groups and a startup
company. This work was
\href{http://www.arl.army.mil/www/default.cfm?article=2462}{recognized
by ARL} as well as the
\href{http://www.army.mil/article/123473/Visiting_Army_scientist_makes_discoveries_in_emerging_technology/}{official
homepage of the US Army}.

\textbf{Carnegie Mellon University} - Postdoctoral Researcher, Prof.
Robert Davis (June 2008 - July 2011).

At CMU I was the lead postdoc on a project to develop a nanolithography
technique using a scanning tunneling microscope tip as a stylus. During
this time I built the \href{https://www.flickr.com/groups/tfan/}{TFAN
nanolithography/surface science laboratory} from scratch. This lab
featured an ultrahigh vacuum surface analysis system including in-situ
sample preparation, scanning probe microscopy, x-ray photoemission
spectroscopy, Auger electron spectroscopy, and low-energy electron
diffraction. I managed three graduate students and we developed
processes to clean and hydrogen passivate Si (100), write features in
the hydrogen passivation with the scanning probe tip, and adsorb
disilane to the features.

\textbf{NC State University} - Graduate Research Assistant, Profs.
Robert J. Nemanich and Griff L. Bilbro (August 2002 - August 2007).

At NCSU I investigated thermionic energy conversion devices based on
diamond electrodes. I developed a comprehensive model of electron
transport through the device which could account for emission from a
diamond cathode featuring negative electron affinity. I implemented the
model initially in MATLAB and later refactored in python. I also
investigated negative electron affinity amorphous carbon films for the
\href{http://www.nasa.gov/mission_pages/ibex/index.html}{Interstellar
Boundary Explorer (IBEX)} mission in collaboration with Lockheed Martin.
I developed a hydrogen passivation process for the amorphous carbon
films and characterized the negative electron affinity with ultraviolet
photoemission spectroscopy. I applied this process to amorphous carbon
facets which were used as part of the IBEX-Lo detector. The IBEX space
probe was launched October 19, 2008.

\textbf{NC State University} - Undergraduate Researcher, Prof. Lubos
Mitas (August 2001 - May 2002).

Monte Carlo computational molecular dynamics.

\section{Education}

\begin{itemize}
\item
  \textbf{Ph.D.} Physics, NC State University, 2007
\item
  \textbf{B.S.} Physics, Cum Laude, NC State University, 2002
\item
  \textbf{B.S.} Mathematics, Cum Laude, NC State University, 2002
\item
  \textbf{High School} North Carolina School of Science and Mathematics,
  1998
\end{itemize}

\section{Teaching}

\begin{itemize}
\item
  \href{https://pages.nist.gov/2015-09-23-nist/}{Software Carpentry
  Workshop, National Institute of Standards and Technology, Gaithersburg
  MD. September 23-24, 2015}
\item
  \href{https://pages.nist.gov/2015-07-23-nist/}{Software Carpentry
  Workshop, National Institute of Standards and Technology, Gaithersburg
  MD. July 23-24, 2015}
\item
  Software Carpentry Bootcamp, Carnegie Mellon University. July 27-28,
  2013
\item
  Software Carpentry Bootcamp, Johns Hopkins University. June 18-19,
  2012
\item
  Software Carpentry Bootcamp, University of Chicago. April 2-3, 2012
\end{itemize}

\section{Service}

\begin{itemize}
\item
  President, Graduate Physics Student Association (GPSA). April 2005
  -April 2006
\item
  University Graduate Student Assc. Representative. April 2003 - April
  2005
\end{itemize}

\section{Publications and presentations}
For a complete list of my publications, please see my online CV at \url{https://goo.gl/roZJ3Q}

